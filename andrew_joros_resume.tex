%-------------------------
% Resume in Latex
% Author : Andrew Joros
% Based off of: https://github.com/sb2nov/resume
% License : MIT
%------------------------

\documentclass[letterpaper,11pt]{article}

\usepackage{latexsym}
\usepackage[empty]{fullpage}
\usepackage{titlesec}
\usepackage{marvosym}
\usepackage[usenames,dvipsnames]{color}
\usepackage{verbatim}
\usepackage{enumitem}
\usepackage[hidelinks]{hyperref}
\usepackage{fancyhdr}
\usepackage[english]{babel}
\usepackage{tabularx}
\input{glyphtounicode}


%----------FONT OPTIONS----------
% sans-serif
% \usepackage[sfdefault]{FiraSans}
% \usepackage[sfdefault]{roboto}
% \usepackage[sfdefault]{noto-sans}
% \usepackage[default]{sourcesanspro}

% serif
% \usepackage{CormorantGaramond}
% \usepackage{charter}


\pagestyle{fancy}
\fancyhf{} % clear all header and footer fields
\fancyfoot{}
\renewcommand{\headrulewidth}{0pt}
\renewcommand{\footrulewidth}{0pt}

% Adjust margins
\addtolength{\oddsidemargin}{-0.5in}
\addtolength{\evensidemargin}{-0.5in}
\addtolength{\textwidth}{1in}
\addtolength{\topmargin}{-.5in}
\addtolength{\textheight}{1.0in}

\urlstyle{same}

\raggedbottom
\raggedright
\setlength{\tabcolsep}{0in}

% Sections formatting
\titleformat{\section}{
  \vspace{-4pt}\scshape\raggedright\large
}{}{0em}{}[\color{black}\titlerule \vspace{-5pt}]

% Ensure that generate pdf is machine readable/ATS parsable
\pdfgentounicode=1

%-------------------------
% Custom commands
\newcommand{\resumeItem}[1]{
  \item\small{
    {#1 \vspace{-2pt}}
  }
}

\newcommand{\resumeSubheading}[4]{
  \vspace{-2pt}\item
    \begin{tabular*}{0.97\textwidth}[t]{l@{\extracolsep{\fill}}r}
      \textbf{#1} & #2 \\
      \textit{\small#3} & \textit{\small #4} \\
    \end{tabular*}\vspace{-7pt}
}

\newcommand{\resumeSubSubheading}[2]{
    \item
    \begin{tabular*}{0.97\textwidth}{l@{\extracolsep{\fill}}r}
      \textit{\small#1} & \textit{\small #2} \\
    \end{tabular*}\vspace{-7pt}
}

\newcommand{\resumeProjectHeading}[2]{
    \item
    \begin{tabular*}{0.97\textwidth}{l@{\extracolsep{\fill}}r}
      \small#1 & #2 \\
    \end{tabular*}\vspace{-7pt}
}

\newcommand{\resumeSubItem}[1]{\resumeItem{#1}\vspace{-4pt}}

\renewcommand\labelitemii{$\vcenter{\hbox{\tiny$\bullet$}}$}

\newcommand{\resumeSubHeadingListStart}{\begin{itemize}[leftmargin=0.15in, label={}]}
\newcommand{\resumeSubHeadingListEnd}{\end{itemize}}
\newcommand{\resumeItemListStart}{\begin{itemize}}
\newcommand{\resumeItemListEnd}{\end{itemize}\vspace{-5pt}}

%-------------------------------------------
%%%%%%  RESUME STARTS HERE  %%%%%%%%%%%%%%%%%%%%%%%%%%%%


\begin{document}

%----------HEADING----------
\begin{center}
    \textbf{\Huge \scshape Andrew Joros} \\ \vspace{1pt}
    \small 2215 Raggio Pkwy, Reno, NV 89512 $|$ 661-607-2503 $|$ 
    \href{mailto:andrew.joros@dri.edu}{\underline{andrew.joros@dri.edu}}
\end{center}


%-----------SUMMARY-----------
\section{Summary}
Highly skilled and experienced data scientist with over 12 years of expertise in cloud computing, data engineering, and front-end and back-end development. Proven ability to successfully implement data-driven solutions, optimize data processing efficiency, and deliver impactful insights for informed decision-making. Possesses a strong track record of working with both earth science and genetic science groups, demonstrating a deep understanding of the unique challenges and opportunities presented by these fields. Currently focused on exploring the potential of AI/GPT technology to revolutionize earth science research. Committed to leveraging cutting-edge technologies to advance our understanding of the natural world and address critical environmental challenges.


%-----------EXPERIENCE-----------
\section{Experience}
  \resumeSubHeadingListStart

    \resumeSubheading
      {Research Computing Engineer (Cloud Computing / Data Scientist)}{2019 -- 2023}
      {Center for Genomic Medicine \& Climate Ecosystem Fire Applications, Desert Research Institute}{Reno, NV}
      \resumeItemListStart
        \resumeItem{Provided cloud computing and data engineering support for genetic analysis projects}
        \resumeItem{Worked with team members on cloud-based data engineering and statistical projects, improving productivity}
        \resumeItem{Led AWS onboarding and training, enhancing the team's cloud computing skills}
        \resumeItem{Created AWS-based solutions for statistical smoke dispersion models for California's fire management}
        \resumeItem{Developed a dynamic front-end webpage using AWS, presenting statistical modeling outputs related to Interagency Wildland Fire Air Quality, empowering stakeholders with valuable insights into air quality conditions}
      \resumeItemListEnd
      
    \resumeSubheading
      {Staff Research Scientist (Software Development / Data Scientist)}{2016 -- 2019}
      {Applied Innovation Center, Desert Research Institute}{Reno, NV}
      \resumeItemListStart
        \resumeItem{Developed water surveying software for groundwater assessment using vertical electrical sounding, enhancing well siting and drilling}
        \resumeItem{Contributed to the construction of meteorological and air pollutant data pipeline, aiding environmental research for the Healthy Nevada Project}
        \resumeItem{Integrated Christopher Ranch's sensor data into AWS for storage and machine learning use}
        \resumeItem{Leveraged AWS EC2 to devise and implement cutting-edge agricultural insect and disease probabilistic models, empowering customers with accurate predictions for effective pest management and crop protection}
        \resumeItem{Created an automated AWS sensor alert system for data integrity and reliability}
      \resumeItemListEnd

    \resumeSubheading
      {Assistant Research Scientist (High Performance Computing)}{2015 -- 2016}
      {Applied Innovation Center, Desert Research Institute}{Reno, NV}
      \resumeItemListStart
        \resumeItem{Constructed backend data architecture for the WINDS platform, improving weather data processing and decision support}
        \resumeItem{Developed a web portal for NDOT, enabling engineers to access and analyze traffic data for transportation planning}
        \resumeItem{Led NevCAN's data visualization project, presenting real-time climate and ecological data effectively}
        \resumeItem{Created an image processing algorithm for greenhouse experiments to analyze vegetation health and growth}
      \resumeItemListEnd

    \resumeSubheading
      {Staff Scientist Programmer, Hydrometeorologist}{2013 -- 2015}
      {Desert Research Institute}{Reno, NV}
      \resumeItemListStart
        \resumeItem{Used Earth Engine JavaScript API for geospatial data analysis in hydrometeorology}
        \resumeItem{Contributed to the Climate and Integrated Earth Monitoring Engine dashboard, enhancing environmental data visualization}
        \resumeItem{Processed Landsat and weather data via cloud computing, aiding Nevada's ecosystem analysis and conservation}
        \resumeItem{Evaluated precipitation datasets against Texas weather stations to improve irrigation and water management}
      \resumeItemListEnd

    \resumeSubheading
      {Research Assistant}{2009 -- 2013}
      {Desert Research Institute}{Reno, NV}
      \resumeItemListStart
        \resumeItem{Investigated monsoonal surges in the Northern Great Basin using statistical and dynamic methods}
        \resumeItem{Improved weather datasets to better represent irrigated agricultural environments}
        \resumeItem{Worked with CEFA to create a RAWS network density plot, aiding land managers' decisions}
        \resumeItem{Developed web frameworks for NOAA-funded drought monitoring and an El Nino climate risk project}
      \resumeItemListEnd

    \resumeSubheading
      {Statistical Analyst}{Summer 2009}
      {Atmospheric Systems Corporation}{Valencia, CA}
      \resumeItemListStart
        \resumeItem{Conducted statistical quality assurance and quality control (QA/QC) of output data from Sonic Detection and Ranging (SODAR) systems}
      \resumeItemListEnd

  \resumeSubHeadingListEnd


%-----------EDUCATION-----------
\section{Education}
  \resumeSubHeadingListStart
    \resumeSubheading
      {University of Nevada}{Reno, NV}
      {M.S. in Atmospheric Science}{2009 -- 2011}
    \resumeSubheading
      {San Jose State University}{San Jose, CA}
      {B.S. in Meteorology}{2006 -- 2009}
  \resumeSubHeadingListEnd


%-----------PROGRAMMING SKILLS-----------
\section{Skills}
 \begin{itemize}[leftmargin=0.15in, label={}]
    \small{\item{
     \textbf{Programming Languages}{: Python, MATLAB, R, Java, C/C++, FORTRAN, Scala} \\
     \textbf{Software Frameworks}{: Apache Hadoop, Apache Spark, Cloudera CDH, ArcGIS} \\
     \textbf{Web Design (Frontend)}{: HTML5, CSS, Adobe Creative Suite, JavaScript, Highcharts} \\
     \textbf{Web Design (Backend)}{: PHP, MySQL, Flask, Django, XAMPP} \\
     \textbf{Cloud Platforms}{: Amazon Web Services, Google Cloud Platform, Azure} \\
     \textbf{Operating Systems}{: Linux (Ubuntu/CentOS/Redhat), Windows}
    }}
 \end{itemize}


%-----------PUBLICATIONS AND REPORTS-----------
\section{Publications and Reports}
 \begin{enumerate}[leftmargin=0.25in, label=\arabic*.]
    \small{\item Kiser, D., Metcalf, W.J., Elhanan, G., Schnieder, B., Schlauch, K., Joros, A., Petersen, C. and Grzymski, J., 2020. Particulate matter and emergency visits for asthma: a time-series study of their association in the presence and absence of wildfire smoke in Reno, Nevada, 2013–2018. Environmental Health, 19(1), pp.1-12.}
    \item Caldwell, T. G., Huntington, J. L., Scanlon, B., Joros, A. N., Howard, T., (2017): Improving Irrigation Water Use Estimates with Remote Sensing Technologies: An Initial Feasibility Study for Texas, 137 p., University of Texas, Bureau of Economic Geology: Austin, TX, Report prepared for Texas Water Development Board.
    \item Joros, A., J. Abatzoglou, N. Nauslar, B. Hatchett, M. Kaplan, (2017) Extratropical Control of Monsoonal Surges in the Great Basin. Monthly Weather Review (In Progress)
    \item Huntington, J. L., Gangopadhyay, S., Spears, M., Allen, R., King, D. L., Morton, C. G., Harrison, A., McEvoy, D. J., Joros, A. N., (2015): West-Wide Climate Risk Assessments: Irrigation Demand and Reservoir Evaporation Projections, Technical memorandum No. 68-68210-2014-01, 218 p, 754 appx, U.S. Bureau of Reclamation
    \item Morton, C. G., Huntington, J. L., Allen, R. G., Kilic, A., Joros, A. N., (2015): More Landsat Satellites Equates to More Reliable Monitoring of Water Consumption, Remote Sensing, Submitted
    \item Joros, A., M. Kaplan, J. Abatzoglou (2011) Extratropical Control of Monsoonal Surges in the Great Basin. (Master's Thesis)
 \end{enumerate}


%-----------CONFERENCE PAPERS/PRESENTATIONS-----------
\section{Conference Papers/Presentations}
 \begin{enumerate}[leftmargin=0.25in, label=\arabic*.]
    \small{\item Rowan, C., Metcalf, W. J., Elhanan, G., Joros, A. N., \& Grzymski, J. J. (2018). Short-Term Sentinel Air Events in Relation to Health Care Utilization for Specific Health Conditions in Reno, Nevada. In B24. AN UPDATE ON INDOOR AND OUTDOOR AIR POLLUTION (pp. A2815-A2815). American Thoracic Society.}
    \item Rosenquist, N. A., Metcalf, J., Elhanan, G., Grzymski, J., Joros, A. N., Field-Ridley, A., \& Darrow, L. A. (2018, August). Acute Associations between PM2. 5 and Allergic and Nonallergic Asthma Exacerbations in Children and Adults. In ISEE Conference Abstracts (Vol. 2018, No. 1).
    \item Hegewisch, K., Daudert, B., Morton, C. G., Peterson, A., Joros, A. N., McEvoy, D. J., Erickson, T., Huntington, J. L., Abatzoglou, J. T. (2015). Google Drought Monitoring Through Cloud Computing and Visualization of Remote Sensing and Meteorological Datasets: Examples for California, American Geophysical Union Chapman Conference: Irvine, CA, April 20, 2015-April 22, 2015
    \item Huntington, J. L., Daudert, B., Morton, C. G., McEvoy, D. J., Joros, A. N., Abatzoglou, J. T., Hegewisch, K., Peterson, A., Vansant, D. S., Allen, R., Kilic, A., Hobbins, M., Verdin, J. (2015). Cloud Computing for Drought Monitoring with Google Earth Engine, U.S. Drought Monitor Forum: Desert Research Institute, Reno, NV, April 14, 2015-April 16, 2015
    \item Huntington, J. L., Gangopadhyay, S., Spears, M., Allen, R., King, D. L., Morton, C. G., Harrison, A., McEvoy, D. J., Joros, A. N. (2015). West-Wide Climate Risk Assessments: Irrigation Demand and Reservoir Evaporation Projections, Technical memorandum No. 68-68210-2014-01, 218 p, 754 appx, U.S. Bureau of Reclamation
    \item Morton, C. G., Huntington, J. L., Allen, R. G., Kilic, A., Joros, A. N. (2015). More Landsat Satellites Equates to More Reliable Monitoring of Water Consumption, Remote Sensing, Submitted
    \item Huntington, J. L., Morton, C. G., McGwire, K. C., Allen, R., Gorelick, N., Thau, D., Joros, A. N. (2014). Monitoring Groundwater Dependent Ecosystems in the Great Basin from Space and Clouds, American Water Resources Association Summer Specialty Conference: Reno, NV, May 31, 0014-July 2, 2014
    \item Huntington, J. L., Morton, C. G., McGwire, K. C., Joros, A. N., Peterson, S., Gorelick, N., Thau, D., Allen, R. (2014). Cloud Computing of Landsat Imagery and Gridded Weather Data for Evaluating Groundwater Dependent Ecosystems in Nevada, Nevada Water Resources Association Annual Conference: Las Vegas, NV, February 2, 2014-February 2, 2014
    \item Huntington, J. L., Morton, C. G., McGwire, K. C., Joros, A. N., Peterson, S., Gorelick, N., Thau, D., Allen, R., 2013: Utilizing Cloud Computing of Landsat Imagery and Gridded Weather Data for Evaluating Groundwater Dependent Ecosystems in Nevada. Nevada Water Resources Association Annual Conference, Las Vegas, NV, 3-6 February 2014
    \item Richard G. Allen, University of Idaho, Kimberly, ID; and J. L. Huntington, A. Kilic, H. Debruin, and A. Joros, (2013): Conditioning of NLDAS, NARR and arid weather station data to improve their representation of well-watered (reference) environments associated with irrigated agriculture. 93rd AMS Annual Meeting, 27th Conference on Hydrology, Austin, TX, 5-10 January 2013
    \item Kelly T. Redmond, DRI, Reno, NV; and J.T. Abatzoglou, D. McEvoy, A. Joros, D. VanSant, and L. M. Edward, (2013): The WestWide Drought Tracker: Drought Monitoring at Fine Spatial Scales 93rd AMS Annual Meeting, Austin, TX, 5-10 January 2013
    \item Nauslar, Nick, J. Abatzoglou, A. Joros (2011): Development of an objective methodology for diagnosing Santa Ana winds. Ninth Symposium on Fire and Forest Meteorology, Palm Springs, CA, 18-20 October 2011
    \item Joros, A., J. F. Mejia (2011): Impact of Eastern and Central Pacific ENSO events over the Continental United States. AMS Applied Climatology Conference 2011, Asheville, North Carolina, 18–20 July 2011
    \item King, K. C., M. L. Kaplan, A. Joros, M. Liddle and E. Uher, (2011): Evaluation of the Operational Multi-scale Environment model with Grid Adaptivity (OMEGA) for use in Wind Energy Potential Assessment in the Great Basin of Nevada. Weather, Climate, and the New Energy Economy - Observations, Modeling, and Data for the Energy Sector. 91st AMS Annual Meeting, Seattle WA, 23-27 January 2010.
    \item Abatzoglou, J.T., K.T. Redmond, L.M. Edwards and A. Joros (2011): Monitoring of drought in the western United States: metrics suitable for tracking drought historically and throughout the 21st century. 91st AMS Annual Meeting, Seattle WA, 23-27 January 2010.
    \item Joros, A., J. Abatzoglou, J. Favors, B. Tan, M. Kaplan (2010): Extratropical Control of Monsoonal Surges into the Great Basin. AMS Mountain Meteorology Conference, Lake Tahoe, CA, 30 August – 3 September 2010.
    \item Tan, B., A. Joros, and M. L. Kaplan, (2010): The Role of Midlatitude Circulations in Triggering Extratropical Convection during the 2004 North American Monsoon. 90th AMS Annual Meeting, Atlanta, GA, 17-22 January 2010.
    \item King, K.C., M. Kaplan, A. Joros, M. Liddle, E. Uher (2010): Evaluation of the Operation Multiscale Model with Grid Adaptivity (OMEGA) for use in Wind Energy Potential Assessment in the Great Basin of Nevada. AMS Mountain Meteorology Conference, Lake Tahoe, CA, 30 August – 3 September 2010.
    \item Uher, Erich, M. Kaplan, A. Joros, D. Decker (2010) Air Pollution Dispersion Forecasting: A Climatological Study of Cape Canaveral Tropospheric Wind Patterns. 90th AMS Annual Meeting, Atlanta, GA, 17-22 January 2010.
 \end{enumerate}


%-------------------------------------------
\end{document}
